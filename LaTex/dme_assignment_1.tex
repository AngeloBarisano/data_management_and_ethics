\documentclass[a4paper]{article}
\newtheorem{hyp}{Hypothesis}
\usepackage{graphicx}
\graphicspath{ {/home/angelo/Documents/Uni/Courses/Data Managment & Ethics/Integrated Assignment/assignemnet_project_folder/ERDs/} }


\begin{document}





\title{DME Integrative Assignment}
\author{Angelo Barisano; 508903 }
\date{September 16th, 2022}
\maketitle

\newpage
\section{Task 1: Plan \& Explore: DEfine Questions for project and familiarize with data}

\subsection{Origin of Data \& Purpose Introduction}
\paragraph{Introduction} Data drives security. The wide spread adoption of data information systems has been used to recognize crime-hotspots to increase policing efficiency and protect people. Thus, creating effective information systems for crime prevention is at the center of policy makers and executive branches of governments. This trend has led to the Chicago PD reaching out to this Data Management Team with the request of creating a Data Management System for crime-data in Chicago.

To this end, the Chicago PD and this Data Management Unit is interested on setting the foundation for a FAIR database; i.e. a Data Base ment to flexible and easy in use of slicing and dicing data to find answers and, contemporaneously, make them available. Additiionally, any code produced durign this endeavour will be made public on github for transparency purposes. 

\paragraph{Data description in scope, volumne, and format}
The data was provided by the Chicago PD contains a sample of apprx. 730,000 registered crimes in the administrative districts of Chicago between the years of 2017 and 2021. The initial data source is in csv-format and pertains to specific recroded crimes in the adminsitrative jurisdiction of the Chicago PD in addition to time, location, crime type, and arrested or not. Thus, the data contact is to be found on the Chicago PD website (\textbf{CITE THE CHICAGO WEBSITE FOR THE DATA}). Considerations regarding the ethical (\& GDPR compliant) use of the data will be discussed in part 5 of this assignment.

\paragraph{Project timeframe, researchers, and misc information} The set project timeframe is the 27th of August 2022 till 1st of October 2022. Involved in this project is only one student, Angelo Barisano. Additionally, this project is designed to comply with FAIR standards (\textbf{CITE FAIR STUFF HERE}).

\subsection{WHAT IS THE PURPOSE OF THIS ASSIGNMENT??}

\subsection{Research Question}
The overarching goal of this project is to set the foundations to funnel the existing crime-data into a data model to improve its readability (analytics), accessibility (logical structure \& metric-interpretability), and further use in feeding systems for advanced predictive crime prevention. 


\textbf{In an iterative (agile) developmentcylce the following questions have been set up: Chicargo is known for its high homocide rate; this leads to policy makers focusing on this issue the most. }

Upon conducting an initial Exploratory Data Analysis (EDA), three major categories could be identified in the existing data: 

\begin{itemize}
  \item Location
  \item Crime type
  \item Crime record data (time, arrest, etc.)
\end{itemize}

Based on the data categories, the EDA, and the aforementioned goal of creating a flexible DB the following research questions guide the creation of the database in increasing complexity. However, the goal is to explore possible combinations of the data by rearranging the dimensions at hand and not to follow a specific question.

\paragraph{Question 1} Finding prevalent crime patterns in the data is a common starting point. Thus, this question provides future research with an adjustable query to gain an overview over the general distribution of crime by type. Subsequently applying a temporal component (i.e. by day of week,  month, year, season, etc.) to this initial distribution reveals trends and temporal patters. This way, this question helps to answer questions to policy makers regarding general trends; such as how crime developed overall and by type. As such, the first question combines the temporal component with the frequency distribution of crime types:

\begin{itemize}
  \item How do crime types (e.g. homocides) patterns distributed by a temporal component? Are trends observable for different crime types?
How do certain crime types (homicide) patterns distibute by seasonality patterns, by time of day? A specific focus will be laid on policy relevant crimes such as homocides.
\end{itemize}



\paragraph{Question 2} The aforementioned question(s) enable users to discern trends for crime types in the data. The next logical progression is to observe the location and crime dimension together. The assumption is that certain districts and beats tend to be more prevalent in cretain crime types. Thus, question two follows:

\begin{itemize}
  \item Do certain crime patterns (e.b. for homocides) persist by location such as certain districts/ beats/ blocks?
\end{itemize}

\paragraph{Question 2} The final step in this analysis is to merge the trend analysis by crime type with the location trends. This analysis by time enables policy makers to discern localized trends in the data in order to address crime patterns by distributing resources more efficiently through the allocation of resources to necessary districts at certain times of the day. In this context, multiple accessory dimensions will be integrated into the analysis to provide a hollistic description of the situation on the ground. For instance assuming that crimes that lead to more arrests are more resoruce intensive, these crimes put a disproportionate strain on law enforcement. Thus, by triangulating arrests by location and e.g. time of day this will enable us to show areas that need more attention by law enforcement. Another angle would be a specific analysis of beats. Beats are the smallest admininistrative unit of a police district; a beat is patrolled for one year by one unit and ten transferred to another beat. Thus, it might be interesting to investigate the connection between a subset of beats that suddenly stop showing problems during one year and then re-appear in terms of crime in antoher year. The subquestion would, thus, investigate whether beats that usually persisted in crime only persist on a closed yearly basis. This way, effective police units might be identified and resources might be allocated more efficiently.
 
\begin{itemize}
  \item Triangualting time, crimetype, and location which areas persit in certain crimes wrt. time?
In order to prevent homocides; which “beats“ are the most prevalent among homcides? During which time of day (for effective allocation of policing resources)? 

\end{itemize}

As such, the final question considers a variety of hypotheses that can be explored. Overall, these project based questions are constructed in such a way that they guide an external user (PD) through the process of finding areas that need successively complex "sliced \ diced" information and culminate in the creation of actionable policy implications regarding prime prevention through resource allocation.


\textbf{ADD PLOTS ALREADY HERE!!!}


\section{Design and Organize: Identify and collect the necessary requirements for the databased based on the questions, data and information needs}
In order to address these questions, the relevant information must be broken down into its underlying entities for easier handling. 

Part 1 identified three borad data categories, which are releveant for this endeavour. 
\paragraph{Entitiy 1: Case} The first component consist of the individual instances of cases themselves populating the database. Conceptually, the instances of cases are central to this project as they enable the creation of the frequency distribution conditional on time and/or location. The variables that define this entity are as follows:

\begin{itemize}
  \item CaseId
  \item DateTime object implicitly containing day, month, year, and time of day 
  \item Arrest Boolean  
´  \item \textbf{Location Description???????????????????????????????????????????}
\item \textbf{Location Description???????????????????????????????????????????}
\item \textbf{Location Description???????????????????????????????????????????}?????????????????????????????????????????????????????
\end{itemize}
 
CaseId is will identify the entity and function as the Primary identifier for each instance further in this analyiss. The DateTime object, though prone notoriously to errors, will enable the clustering of crimes by the time dimension. Additionally, arrest information is used to further drill down the analysis and disect the cases for more resource intensive cases. Finally, the dimensions of Latitude and Longitude will be included on the case level. The reason for this is, while lat-lon information pertains to the location category, this variable(s) is case specific due to its precision. Imagine the database where latitude and longitude would be descirbed as a standalone entity hierarchically below district or beat. In order to make the connection to each individual case, each location would have to carry a CaseId as a foreign key in order to identify what type of crime was perpetrated at this location. Assuming that the lat-lon information is sufficiently granular, this would create a one-to-one relationship between case and the lat-lon entity (disregarding district in this case). Any one to one relationship is redundant by definition. Thus, it does not make sense to place lat-lon as a standalone entity hierarchically below district. The reason for the inclusion of lat-lon is its usage as geopandas objects in plotting crime in a heatmap in part 6. Please note that while the information of lat-lon technically violates 1st NF requirements, this information is not supposed to be processed via SQL (except for PostGIS applications). Lat-lon information in tuple form is usually interpreted as a dataobject in other scripting langages and is, thus, just one atomic item by definition.

\textbf{The main function of case as an entitiy is to make the following two dimenstions compliant to the 2nd normal form (and to some extend compliant with 3rd normal form with some caveats) by default. If we were to leave out case as an entity and immediately match crime types and location, the resulting two to three entities could not comply with 3rd normal form as crime types and location (as categories) would produce a many-to-many relationship. As such, the case entitiy inatvertently functions as a line item relation, reducing many-to-many relationships to two many-to-one relationships.\footnote{I will not further elaborate on this; this is a logical conclusion; in order to reach 2nd normal form this is a required step which is obvious} As a consequence, case complies to the 3rd NF by definition as no individual stand alone entities are to be found in this entity, while requiring the other data categories to normalize as well. It is notable, that while such design choices should be reflected by leaving out case as an entity in the conceptual model in part 3 and then include it as part of the logical model, but case is so pivotal to the functioning of the database in its purpose, that we will consider it along the way a a valid entity.}

\paragraph{Entitiy 2: Crime Type} Every Case requires a crime to be a valid instance. The raw data provides the IUCR, which identifies each unique combination of primary (e.g. homocide) and secondary crime category (e.g. first degree). Generally, in most headlines, only consider the primary type of crime, such as homocides, and generally disregard the secondary description of the data, such as first, second, or third degree in this case \textbf{CITE SOME EXAMPLE NEWSPAPERS}. This way, the complexity of the analysis is reduced (from 343 to \textbf{34} crime types)in favour of making the analysis understandable, the questions posed in part 1 can be answered, while no crime instance in entitiy one is dropped. Thus, the secondary crime type becomes superflous and consequently the IUCR as well as an identifier. Only the primary crime type is being considered in this entity. This way, the resulting entity complies by definition with the 3rd NF and the principle of data minimization is complied by. 

\paragraph{Entitiy 3 \& 4: Location} 
The first entity identified in this case pertains to (police) district. Following hierachrically, beats are settled below district; one beat containing multiple beats. 
Thus, a natural hierachie is being identified, resolving any issues regarding normalization when connecting case to district and then to beat. While there is an implicit hierarchy from District to Beat, which translates naturally into the 3rd normal form, a foreign key from both district and beat are required to be included in the case entity. The reason is twofold. 1) If we were to relate any case to beat without the relation of beat with case, both beat and district would no longer comply with the 3rd NF as we would have to create one case instance in district and beat for every record in case or repeat beat information in case. 2) Contrarily, if we were to include beat as a column in case, case as an entity would only comply to the 2nd NF. Moreover, by including beat hierarchically below district (district FK in beat), while maintaining the relationship of beat and case (beat FK in case), any rank based operations (window functions) as by question two and three in part 1 become considerably more efficient and faster. Finally, this design avoids any circular references.

The reason to exclude block as a entity of analysis is threefold. 1) blocks overlap to some extend with different districts and beats increasing the level of analysis unnecessarily. 2) Beats and districts are administrative units, while blocks require local knowledge. 3) The information regarding the location of the case happening is already included in the lat-lon information in entitiy one as block only pertains to a reducted version of the address making the lat-lon information more informative when plotted (also: no map regarding blocks in chicago is available). 


\subsection{WHAT VARIABLES SHOULD BE INCLUDED???}
\textbf{I DO NOT UNDERSTAND THIS PART? ALL INFORMATION IS SOMEWHAT RELEVANT; MAYBE NO BLOCK INFORMATION; BUT MAYBE IT IS RELEVANT. ADDITIONALLY, for task 3, there are no MANY TO MANY relationships to begin with! So what do they want in aprt 3 from me? WHAT DO YOU MEAN WITH VARIABLES HERE?}


\section{Part 3- creation of an entitiy relationship diagram}


\includegraphics[scale=.35 ]{Conceptual_Model_1.png}


\section{Part 4- Data Quality checks and preparation}
DATA TRIANGLE in lecture!!!



\end{document}
